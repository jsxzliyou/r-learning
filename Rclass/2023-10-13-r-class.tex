% Options for packages loaded elsewhere
\PassOptionsToPackage{unicode}{hyperref}
\PassOptionsToPackage{hyphens}{url}
%
\documentclass[
]{article}
\usepackage{amsmath,amssymb}
\usepackage{iftex}
\ifPDFTeX
  \usepackage[T1]{fontenc}
  \usepackage[utf8]{inputenc}
  \usepackage{textcomp} % provide euro and other symbols
\else % if luatex or xetex
  \usepackage{unicode-math} % this also loads fontspec
  \defaultfontfeatures{Scale=MatchLowercase}
  \defaultfontfeatures[\rmfamily]{Ligatures=TeX,Scale=1}
\fi
\usepackage{lmodern}
\ifPDFTeX\else
  % xetex/luatex font selection
\fi
% Use upquote if available, for straight quotes in verbatim environments
\IfFileExists{upquote.sty}{\usepackage{upquote}}{}
\IfFileExists{microtype.sty}{% use microtype if available
  \usepackage[]{microtype}
  \UseMicrotypeSet[protrusion]{basicmath} % disable protrusion for tt fonts
}{}
\makeatletter
\@ifundefined{KOMAClassName}{% if non-KOMA class
  \IfFileExists{parskip.sty}{%
    \usepackage{parskip}
  }{% else
    \setlength{\parindent}{0pt}
    \setlength{\parskip}{6pt plus 2pt minus 1pt}}
}{% if KOMA class
  \KOMAoptions{parskip=half}}
\makeatother
\usepackage{xcolor}
\usepackage[margin=1in]{geometry}
\usepackage{color}
\usepackage{fancyvrb}
\newcommand{\VerbBar}{|}
\newcommand{\VERB}{\Verb[commandchars=\\\{\}]}
\DefineVerbatimEnvironment{Highlighting}{Verbatim}{commandchars=\\\{\}}
% Add ',fontsize=\small' for more characters per line
\usepackage{framed}
\definecolor{shadecolor}{RGB}{248,248,248}
\newenvironment{Shaded}{\begin{snugshade}}{\end{snugshade}}
\newcommand{\AlertTok}[1]{\textcolor[rgb]{0.94,0.16,0.16}{#1}}
\newcommand{\AnnotationTok}[1]{\textcolor[rgb]{0.56,0.35,0.01}{\textbf{\textit{#1}}}}
\newcommand{\AttributeTok}[1]{\textcolor[rgb]{0.13,0.29,0.53}{#1}}
\newcommand{\BaseNTok}[1]{\textcolor[rgb]{0.00,0.00,0.81}{#1}}
\newcommand{\BuiltInTok}[1]{#1}
\newcommand{\CharTok}[1]{\textcolor[rgb]{0.31,0.60,0.02}{#1}}
\newcommand{\CommentTok}[1]{\textcolor[rgb]{0.56,0.35,0.01}{\textit{#1}}}
\newcommand{\CommentVarTok}[1]{\textcolor[rgb]{0.56,0.35,0.01}{\textbf{\textit{#1}}}}
\newcommand{\ConstantTok}[1]{\textcolor[rgb]{0.56,0.35,0.01}{#1}}
\newcommand{\ControlFlowTok}[1]{\textcolor[rgb]{0.13,0.29,0.53}{\textbf{#1}}}
\newcommand{\DataTypeTok}[1]{\textcolor[rgb]{0.13,0.29,0.53}{#1}}
\newcommand{\DecValTok}[1]{\textcolor[rgb]{0.00,0.00,0.81}{#1}}
\newcommand{\DocumentationTok}[1]{\textcolor[rgb]{0.56,0.35,0.01}{\textbf{\textit{#1}}}}
\newcommand{\ErrorTok}[1]{\textcolor[rgb]{0.64,0.00,0.00}{\textbf{#1}}}
\newcommand{\ExtensionTok}[1]{#1}
\newcommand{\FloatTok}[1]{\textcolor[rgb]{0.00,0.00,0.81}{#1}}
\newcommand{\FunctionTok}[1]{\textcolor[rgb]{0.13,0.29,0.53}{\textbf{#1}}}
\newcommand{\ImportTok}[1]{#1}
\newcommand{\InformationTok}[1]{\textcolor[rgb]{0.56,0.35,0.01}{\textbf{\textit{#1}}}}
\newcommand{\KeywordTok}[1]{\textcolor[rgb]{0.13,0.29,0.53}{\textbf{#1}}}
\newcommand{\NormalTok}[1]{#1}
\newcommand{\OperatorTok}[1]{\textcolor[rgb]{0.81,0.36,0.00}{\textbf{#1}}}
\newcommand{\OtherTok}[1]{\textcolor[rgb]{0.56,0.35,0.01}{#1}}
\newcommand{\PreprocessorTok}[1]{\textcolor[rgb]{0.56,0.35,0.01}{\textit{#1}}}
\newcommand{\RegionMarkerTok}[1]{#1}
\newcommand{\SpecialCharTok}[1]{\textcolor[rgb]{0.81,0.36,0.00}{\textbf{#1}}}
\newcommand{\SpecialStringTok}[1]{\textcolor[rgb]{0.31,0.60,0.02}{#1}}
\newcommand{\StringTok}[1]{\textcolor[rgb]{0.31,0.60,0.02}{#1}}
\newcommand{\VariableTok}[1]{\textcolor[rgb]{0.00,0.00,0.00}{#1}}
\newcommand{\VerbatimStringTok}[1]{\textcolor[rgb]{0.31,0.60,0.02}{#1}}
\newcommand{\WarningTok}[1]{\textcolor[rgb]{0.56,0.35,0.01}{\textbf{\textit{#1}}}}
\usepackage{graphicx}
\makeatletter
\def\maxwidth{\ifdim\Gin@nat@width>\linewidth\linewidth\else\Gin@nat@width\fi}
\def\maxheight{\ifdim\Gin@nat@height>\textheight\textheight\else\Gin@nat@height\fi}
\makeatother
% Scale images if necessary, so that they will not overflow the page
% margins by default, and it is still possible to overwrite the defaults
% using explicit options in \includegraphics[width, height, ...]{}
\setkeys{Gin}{width=\maxwidth,height=\maxheight,keepaspectratio}
% Set default figure placement to htbp
\makeatletter
\def\fps@figure{htbp}
\makeatother
\setlength{\emergencystretch}{3em} % prevent overfull lines
\providecommand{\tightlist}{%
  \setlength{\itemsep}{0pt}\setlength{\parskip}{0pt}}
\setcounter{secnumdepth}{-\maxdimen} % remove section numbering
\ifLuaTeX
  \usepackage{selnolig}  % disable illegal ligatures
\fi
\IfFileExists{bookmark.sty}{\usepackage{bookmark}}{\usepackage{hyperref}}
\IfFileExists{xurl.sty}{\usepackage{xurl}}{} % add URL line breaks if available
\urlstyle{same}
\hypersetup{
  pdftitle={r\_class\_2023-10-14},
  pdfauthor={liyouyou},
  hidelinks,
  pdfcreator={LaTeX via pandoc}}

\title{r\_class\_2023-10-14}
\author{liyouyou}
\date{2023-10-14}

\begin{document}
\maketitle

\hypertarget{function-ux51fdux6570}{%
\section{Function 函数}\label{function-ux51fdux6570}}

\begin{Shaded}
\begin{Highlighting}[]
\CommentTok{\# 平方和}
\NormalTok{square\_plus }\OtherTok{\textless{}{-}} \ControlFlowTok{function}\NormalTok{(x) \{}
\NormalTok{  squ }\OtherTok{\textless{}{-}}\NormalTok{ x }\SpecialCharTok{*}\NormalTok{ x}
  \FunctionTok{sum}\NormalTok{(squ, }\AttributeTok{na.rm =} \ConstantTok{TRUE}\NormalTok{)}
\NormalTok{\}}
\FunctionTok{square\_plus}\NormalTok{(}\FunctionTok{c}\NormalTok{(}\DecValTok{1}\NormalTok{,}\DecValTok{2}\NormalTok{,}\DecValTok{3}\NormalTok{,}\DecValTok{4}\NormalTok{,}\DecValTok{5}\NormalTok{,}\DecValTok{6}\NormalTok{,}\DecValTok{7}\NormalTok{,}\DecValTok{8}\NormalTok{,}\DecValTok{9}\NormalTok{,}\DecValTok{10}\NormalTok{))}
\end{Highlighting}
\end{Shaded}

\begin{verbatim}
## [1] 385
\end{verbatim}

\begin{Shaded}
\begin{Highlighting}[]
\CommentTok{\# 原始1}
\NormalTok{square\_plus2 }\OtherTok{\textless{}{-}} \ControlFlowTok{function}\NormalTok{(x) \{}
\NormalTok{  out }\OtherTok{\textless{}{-}} \FunctionTok{vector}\NormalTok{(}\StringTok{"integer"}\NormalTok{, }\FunctionTok{length}\NormalTok{(x))}
  \ControlFlowTok{for}\NormalTok{ (i }\ControlFlowTok{in} \FunctionTok{seq\_along}\NormalTok{(x)) \{}
\NormalTok{    out[[i]] }\OtherTok{\textless{}{-}}\NormalTok{ x[[i]] }\SpecialCharTok{*}\NormalTok{ x[[i]]}
\NormalTok{  \}}
  \FunctionTok{sum}\NormalTok{(out)}
\NormalTok{\}}
\FunctionTok{square\_plus2}\NormalTok{(}\FunctionTok{c}\NormalTok{(}\DecValTok{1}\NormalTok{,}\DecValTok{2}\NormalTok{,}\DecValTok{3}\NormalTok{,}\DecValTok{4}\NormalTok{,}\DecValTok{5}\NormalTok{,}\DecValTok{6}\NormalTok{,}\DecValTok{7}\NormalTok{,}\DecValTok{8}\NormalTok{,}\DecValTok{9}\NormalTok{,}\DecValTok{10}\NormalTok{))}
\end{Highlighting}
\end{Shaded}

\begin{verbatim}
## [1] 385
\end{verbatim}

\begin{Shaded}
\begin{Highlighting}[]
\CommentTok{\# 原始2}
\NormalTok{square\_plus3 }\OtherTok{\textless{}{-}} \ControlFlowTok{function}\NormalTok{(x) \{}
\NormalTok{  out }\OtherTok{\textless{}{-}} \DecValTok{0}
  \ControlFlowTok{for}\NormalTok{ (i }\ControlFlowTok{in} \FunctionTok{seq\_along}\NormalTok{(x)) \{}
\NormalTok{    out }\OtherTok{\textless{}{-}}\NormalTok{ out }\SpecialCharTok{+}\NormalTok{ x[i] }\SpecialCharTok{\^{}} \DecValTok{2}
\NormalTok{  \}}
  \FunctionTok{return}\NormalTok{ (out)}
\NormalTok{\}}
\FunctionTok{square\_plus3}\NormalTok{(}\FunctionTok{c}\NormalTok{(}\DecValTok{10}\NormalTok{, }\DecValTok{48}\NormalTok{, }\DecValTok{20}\NormalTok{))}
\end{Highlighting}
\end{Shaded}

\begin{verbatim}
## [1] 2804
\end{verbatim}

\begin{Shaded}
\begin{Highlighting}[]
\CommentTok{\# 连续自然数求和}
\NormalTok{conti\_sum }\OtherTok{\textless{}{-}} \ControlFlowTok{function}\NormalTok{(x, y) \{}
  \ControlFlowTok{if}\NormalTok{ (y }\SpecialCharTok{\textless{}}\NormalTok{ x) \{}
    \FunctionTok{return}\NormalTok{ (}\DecValTok{0}\NormalTok{)}
\NormalTok{  \}}
\NormalTok{  sum }\OtherTok{\textless{}{-}} \DecValTok{0}
  \ControlFlowTok{for}\NormalTok{ (i }\ControlFlowTok{in}\NormalTok{ x}\SpecialCharTok{:}\NormalTok{y) \{}
\NormalTok{    sum }\OtherTok{=}\NormalTok{ sum }\SpecialCharTok{+}\NormalTok{ i}
\NormalTok{  \}}
\NormalTok{  sum}
\NormalTok{\}}
\FunctionTok{conti\_sum}\NormalTok{(}\DecValTok{75}\NormalTok{, }\DecValTok{100}\NormalTok{)}
\end{Highlighting}
\end{Shaded}

\begin{verbatim}
## [1] 2275
\end{verbatim}

\begin{Shaded}
\begin{Highlighting}[]
\NormalTok{x }\OtherTok{\textless{}{-}} \FunctionTok{rchisq}\NormalTok{(}\DecValTok{15}\NormalTok{, }\DecValTok{8}\NormalTok{)}
\NormalTok{y }\OtherTok{\textless{}{-}}\NormalTok{ (x }\SpecialCharTok{/} \DecValTok{10}\NormalTok{) }\SpecialCharTok{\^{}} \DecValTok{2} \SpecialCharTok{+}\NormalTok{ x}
\FunctionTok{hist}\NormalTok{(y)}
\end{Highlighting}
\end{Shaded}

\includegraphics{2023-10-13-r-class_files/figure-latex/unnamed-chunk-5-1.pdf}

\begin{Shaded}
\begin{Highlighting}[]
\NormalTok{z }\OtherTok{\textless{}{-}}\NormalTok{ y }\SpecialCharTok{*} \DecValTok{10} \SpecialCharTok{+} \DecValTok{5}
\FunctionTok{ggplot}\NormalTok{(}\FunctionTok{data.frame}\NormalTok{(}\AttributeTok{x =}\NormalTok{ z), }\FunctionTok{aes}\NormalTok{(x)) }\SpecialCharTok{+}
  \FunctionTok{geom\_histogram}\NormalTok{() }
\end{Highlighting}
\end{Shaded}

\begin{verbatim}
## `stat_bin()` using `bins = 30`. Pick better value with `binwidth`.
\end{verbatim}

\includegraphics{2023-10-13-r-class_files/figure-latex/unnamed-chunk-6-1.pdf}

\begin{Shaded}
\begin{Highlighting}[]
\NormalTok{norm\_di }\OtherTok{\textless{}{-}} \ControlFlowTok{function}\NormalTok{(times) \{}
\NormalTok{  result }\OtherTok{\textless{}{-}} \FunctionTok{numeric}\NormalTok{(times)}
  \ControlFlowTok{for}\NormalTok{ (i }\ControlFlowTok{in} \DecValTok{1}\SpecialCharTok{:}\NormalTok{times) \{}
\NormalTok{    x }\OtherTok{\textless{}{-}} \FunctionTok{rchisq}\NormalTok{(}\DecValTok{15}\NormalTok{, }\DecValTok{8}\NormalTok{)}
\NormalTok{    y }\OtherTok{\textless{}{-}}\NormalTok{ (x }\SpecialCharTok{/} \DecValTok{10}\NormalTok{) }\SpecialCharTok{\^{}} \DecValTok{2} \SpecialCharTok{+}\NormalTok{ x}
\NormalTok{    result[i] }\OtherTok{\textless{}{-}} \FunctionTok{mean}\NormalTok{(y)}
\NormalTok{  \}}
\NormalTok{  result}
\NormalTok{\}}
\NormalTok{res }\OtherTok{\textless{}{-}} \FunctionTok{norm\_di}\NormalTok{(}\DecValTok{100000}\NormalTok{)}

\CommentTok{\#hist(res)}
\CommentTok{\#lines(density(res), col = "blue", lwd = 2)}

\FunctionTok{ggplot}\NormalTok{(}\FunctionTok{data.frame}\NormalTok{(}\AttributeTok{x =}\NormalTok{ res), }\FunctionTok{aes}\NormalTok{(x)) }\SpecialCharTok{+} 
  \FunctionTok{geom\_histogram}\NormalTok{(}\AttributeTok{binwidth =} \FloatTok{0.2}\NormalTok{) }\SpecialCharTok{+}
  \FunctionTok{geom\_density}\NormalTok{(}\AttributeTok{color =} \StringTok{"red"}\NormalTok{)}
\end{Highlighting}
\end{Shaded}

\includegraphics{2023-10-13-r-class_files/figure-latex/unnamed-chunk-7-1.pdf}
\#\# 计算pi

\begin{Shaded}
\begin{Highlighting}[]
\NormalTok{times }\OtherTok{\textless{}{-}} \DecValTok{10000000}
\NormalTok{x }\OtherTok{\textless{}{-}} \FunctionTok{runif}\NormalTok{(times, }\AttributeTok{min =} \SpecialCharTok{{-}}\DecValTok{1}\NormalTok{, }\AttributeTok{max =} \DecValTok{1}\NormalTok{)}
\NormalTok{y }\OtherTok{\textless{}{-}} \FunctionTok{runif}\NormalTok{(times, }\AttributeTok{min =} \SpecialCharTok{{-}}\DecValTok{1}\NormalTok{, }\AttributeTok{max =} \DecValTok{1}\NormalTok{)}

\CommentTok{\# 不用for}
\NormalTok{count }\OtherTok{\textless{}{-}} \FunctionTok{sum}\NormalTok{(x}\SpecialCharTok{\^{}}\DecValTok{2} \SpecialCharTok{+}\NormalTok{ y}\SpecialCharTok{\^{}}\DecValTok{2} \SpecialCharTok{\textless{}=} \DecValTok{1}\NormalTok{)}
\DecValTok{4} \SpecialCharTok{*}\NormalTok{ count }\SpecialCharTok{/}\NormalTok{ times}
\end{Highlighting}
\end{Shaded}

\begin{verbatim}
## [1] 3.14159
\end{verbatim}

\begin{Shaded}
\begin{Highlighting}[]
\CommentTok{\# 用for}
\NormalTok{count }\OtherTok{\textless{}{-}} \DecValTok{0}
\ControlFlowTok{for}\NormalTok{ (i }\ControlFlowTok{in} \DecValTok{1}\SpecialCharTok{:}\NormalTok{times) \{}
  \ControlFlowTok{if}\NormalTok{ (x[i] }\SpecialCharTok{\^{}} \DecValTok{2} \SpecialCharTok{+}\NormalTok{ y[i] }\SpecialCharTok{\^{}} \DecValTok{2} \SpecialCharTok{\textless{}=} \DecValTok{1}\NormalTok{) \{}
\NormalTok{    count }\OtherTok{\textless{}{-}}\NormalTok{ count }\SpecialCharTok{+} \DecValTok{1}
\NormalTok{  \}}
\NormalTok{\}}
\DecValTok{4} \SpecialCharTok{*}\NormalTok{ count }\SpecialCharTok{/}\NormalTok{ times}
\end{Highlighting}
\end{Shaded}

\begin{verbatim}
## [1] 3.14159
\end{verbatim}

\hypertarget{ux8ba1ux7b97pi}{%
\subsection{计算pi}\label{ux8ba1ux7b97pi}}

\begin{Shaded}
\begin{Highlighting}[]
\NormalTok{pi }\OtherTok{\textless{}{-}} \ControlFlowTok{function}\NormalTok{(n) \{}
\NormalTok{  result }\OtherTok{\textless{}{-}} \FunctionTok{numeric}\NormalTok{(n)}
\NormalTok{  times }\OtherTok{\textless{}{-}} \DecValTok{5}
  \ControlFlowTok{for}\NormalTok{ (i }\ControlFlowTok{in} \DecValTok{1}\SpecialCharTok{:}\NormalTok{n) \{}
\NormalTok{    x }\OtherTok{\textless{}{-}} \FunctionTok{runif}\NormalTok{(times, }\AttributeTok{min =} \SpecialCharTok{{-}}\DecValTok{1}\NormalTok{, }\AttributeTok{max =} \DecValTok{1}\NormalTok{)}
\NormalTok{    y }\OtherTok{\textless{}{-}} \FunctionTok{runif}\NormalTok{(times, }\AttributeTok{min =} \SpecialCharTok{{-}}\DecValTok{1}\NormalTok{, }\AttributeTok{max =} \DecValTok{1}\NormalTok{)}
\NormalTok{    count }\OtherTok{\textless{}{-}} \FunctionTok{sum}\NormalTok{(x}\SpecialCharTok{\^{}}\DecValTok{2} \SpecialCharTok{+}\NormalTok{ y}\SpecialCharTok{\^{}}\DecValTok{2} \SpecialCharTok{\textless{}=} \DecValTok{1}\NormalTok{)}
\NormalTok{    result[i] }\OtherTok{\textless{}{-}} \DecValTok{4} \SpecialCharTok{*}\NormalTok{ count }\SpecialCharTok{/}\NormalTok{ times}
\NormalTok{  \}}
  \FunctionTok{return}\NormalTok{ (result)}
\NormalTok{\}}
\NormalTok{pi\_hist }\OtherTok{\textless{}{-}} \FunctionTok{pi}\NormalTok{(}\DecValTok{100000}\NormalTok{)}
\FunctionTok{mean}\NormalTok{(pi\_hist)}
\end{Highlighting}
\end{Shaded}

\begin{verbatim}
## [1] 3.14032
\end{verbatim}

\begin{Shaded}
\begin{Highlighting}[]
\FunctionTok{hist}\NormalTok{(pi\_hist)}
\FunctionTok{abline}\NormalTok{(}\AttributeTok{v =} \FunctionTok{mean}\NormalTok{(pi\_hist), }\AttributeTok{col =} \StringTok{"red"}\NormalTok{, }\AttributeTok{lwd =} \DecValTok{1}\NormalTok{) }
\end{Highlighting}
\end{Shaded}

\includegraphics{2023-10-13-r-class_files/figure-latex/unnamed-chunk-9-1.pdf}
\#\# 一堆数,大于均值 除以2 小于均值 乘以2

\begin{Shaded}
\begin{Highlighting}[]
\NormalTok{nor }\OtherTok{\textless{}{-}} \ControlFlowTok{function}\NormalTok{(x) \{}
\NormalTok{  mean\_x }\OtherTok{\textless{}{-}} \FunctionTok{mean}\NormalTok{(x)}
  \ControlFlowTok{for}\NormalTok{ (i }\ControlFlowTok{in} \DecValTok{1}\SpecialCharTok{:}\FunctionTok{length}\NormalTok{(x)) \{}
    \ControlFlowTok{if}\NormalTok{ (x[i] }\SpecialCharTok{\textless{}}\NormalTok{ mean\_x) \{}
\NormalTok{      x[i] }\OtherTok{\textless{}{-}}\NormalTok{ x[i] }\SpecialCharTok{*} \DecValTok{2}
\NormalTok{    \} }\ControlFlowTok{else} \ControlFlowTok{if}\NormalTok{ (x[i] }\SpecialCharTok{\textgreater{}}\NormalTok{ mean\_x) \{}
\NormalTok{      x[i] }\OtherTok{\textless{}{-}}\NormalTok{ x[i] }\SpecialCharTok{/} \DecValTok{2}
\NormalTok{    \} }\ControlFlowTok{else}\NormalTok{ \{}
\NormalTok{      x[i] }\OtherTok{\textless{}{-}}\NormalTok{ x[i] }\SpecialCharTok{*} \FloatTok{1.5}
\NormalTok{    \}}
\NormalTok{  \}}
  \FunctionTok{return}\NormalTok{ (x)}
\NormalTok{\}}
\NormalTok{x }\OtherTok{\textless{}{-}} \FunctionTok{c}\NormalTok{(}\DecValTok{1}\NormalTok{, }\DecValTok{1}\NormalTok{, }\SpecialCharTok{{-}}\DecValTok{1}\NormalTok{, }\SpecialCharTok{{-}}\DecValTok{1}\NormalTok{, }\DecValTok{1}\NormalTok{, }\SpecialCharTok{{-}}\DecValTok{1}\NormalTok{, }\DecValTok{0}\NormalTok{)}
\FunctionTok{nor}\NormalTok{(x)}
\end{Highlighting}
\end{Shaded}

\begin{verbatim}
## [1]  0.5  0.5 -2.0 -2.0  0.5 -2.0  0.0
\end{verbatim}

\hypertarget{r-markdown}{%
\subsection{R Markdown}\label{r-markdown}}

This is an R Markdown document. Markdown is a simple formatting syntax
for authoring HTML, PDF, and MS Word documents. For more details on
using R Markdown see \url{http://rmarkdown.rstudio.com}.

When you click the \textbf{Knit} button a document will be generated
that includes both content as well as the output of any embedded R code
chunks within the document. You can embed an R code chunk like this:

\begin{Shaded}
\begin{Highlighting}[]
\FunctionTok{summary}\NormalTok{(cars)}
\end{Highlighting}
\end{Shaded}

\begin{verbatim}
##      speed           dist       
##  Min.   : 4.0   Min.   :  2.00  
##  1st Qu.:12.0   1st Qu.: 26.00  
##  Median :15.0   Median : 36.00  
##  Mean   :15.4   Mean   : 42.98  
##  3rd Qu.:19.0   3rd Qu.: 56.00  
##  Max.   :25.0   Max.   :120.00
\end{verbatim}

\hypertarget{including-plots}{%
\subsection{Including Plots}\label{including-plots}}

You can also embed plots, for example:

\includegraphics{2023-10-13-r-class_files/figure-latex/pressure-1.pdf}

Note that the \texttt{echo\ =\ FALSE} parameter was added to the code
chunk to prevent printing of the R code that generated the plot.

\end{document}
